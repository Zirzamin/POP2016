%%%%%%%%%%%%%%%%%%%%%%%%%%%%%%%%%%%%%%%%%
% Programming/Coding Assignment
% LaTeX Template
%
% This template has been downloaded from:
% http://www.latextemplates.com
%
% Original author:
% Ted Pavlic (http://www.tedpavlic.com)
%
% Note:
% The \lipsum[#] commands throughout this template generate dummy text
% to fill the template out. These commands should all be removed when 
% writing assignment content.
%
% This template uses a Perl script as an example snippet of code, most other
% languages are also usable. Configure them in the "CODE INCLUSION 
% CONFIGURATION" section.
%
%%%%%%%%%%%%%%%%%%%%%%%%%%%%%%%%%%%%%%%%%

%----------------------------------------------------------------------------------------
%	PACKAGES AND OTHER DOCUMENT CONFIGURATIONS
%----------------------------------------------------------------------------------------

\documentclass{article}

\usepackage[utf8]{inputenc} %For æ ø å and other danish symbols
\usepackage{fancyhdr} % Required for custom headers
\usepackage{lastpage} % Required to determine the last page for the footer
\usepackage{extramarks} % Required for headers and footers
\usepackage[usenames,dvipsnames]{color} % Required for custom colors
\usepackage{graphicx} % Required to insert images
\usepackage{listings} % Required for insertion of code
\usepackage[]{algorithm2e} % For algortihms 
\usepackage{courier} % Required for the courier font
\usepackage{lipsum} % Used for inserting dummy 'Lorem ipsum' text into the template
\usepackage{color}
\usepackage{amsmath}
\usepackage{amssymb}
\usepackage{amsthm}
\usepackage{upquote,textcomp}

\usepackage[hidelinks]{hyperref} % For URL ref
\usepackage{xcolor}
\hypersetup{
	colorlinks,
	linkcolor={red!50!black},
	citecolor={blue!50!black},
	urlcolor={blue!80!black}
}

\graphicspath{ {images/} } %all images are in the folder images

% Margins
\topmargin=-0.45in
\evensidemargin=0in
\oddsidemargin=0in
\textwidth=6.5in
\textheight=9.0in
\headsep=0.25in

\linespread{1.1} % Line spacing

% Set up the header and footer
\pagestyle{fancy}
%\lhead{\hmwkAuthorNameMehr} % Top left header
\chead{\hmwkClass: \hmwkTitle} % Top center head
\rhead{\firstxmark} % Top right header
\lfoot{\lastxmark} % Bottom left footer
\cfoot{} % Bottom center footer
\rfoot{Page\ \thepage\ of\ \protect\pageref{LastPage}} % Bottom right footer
\renewcommand\headrulewidth{0.4pt} % Size of the header rule
\renewcommand\footrulewidth{0.4pt} % Size of the footer rule

\setlength\parindent{0pt} % Removes all indentation from paragraphs

%----------------------------------------------------------------------------------------
%	DOCUMENT STRUCTURE COMMANDS
%	Skip this unless you know what you're doing
%----------------------------------------------------------------------------------------

% Header and footer for when a page split occurs within a problem environment
\newcommand{\enterProblemHeader}[1]{
\nobreak\extramarks{#1}{#1 continued on next page\ldots}\nobreak
\nobreak\extramarks{#1 (continued)}{#1 continued on next page\ldots}\nobreak
}

% Header and footer for when a page split occurs between problem environments
\newcommand{\exitProblemHeader}[1]{
\nobreak\extramarks{#1 (continued)}{#1 continued on next page\ldots}\nobreak
\nobreak\extramarks{#1}{}\nobreak
}

\setcounter{secnumdepth}{0} % Removes default section numbers
\newcounter{homeworkProblemCounter} % Creates a counter to keep track of the number of problems

\newcommand{\homeworkProblemName}{}
\newenvironment{homeworkProblem}[1][1.8 Exercises \arabic{homeworkProblemCounter}]{ % Makes a new environment called homeworkProblem which takes 1 argument (custom name) but the default is "Problem #"
\stepcounter{homeworkProblemCounter} % Increase counter for number of problems
\renewcommand{\homeworkProblemName}{#1-1} % Assign \homeworkProblemName the name of the problem
\section{\homeworkProblemName} % Make a section in the document with the custom problem count
\enterProblemHeader{\homeworkProblemName} % Header and footer within the environment
}{
\exitProblemHeader{\homeworkProblemName} % Header and footer after the environment
}

%----------------------------------------------------------------------------------------
%	NAME AND CLASS SECTION
%----------------------------------------------------------------------------------------

\newcommand{\hmwkTitle}{Ugeseddel\ \#5} % Assignment title
\newcommand{\hmwkDueDate}{Wednesday,\ October\ 12,\ 2016} % Due date
\newcommand{\hmwkClass}{Programmering og problemløsning 5100-B1-2E16} % Course/class
%\newcommand{\hmwkClassTime}{09:15am} % Class/lecture time
%\newcommand{\hmwkClassInstructor}{Jones} % Teacher/lecturer
\newcommand{\hmwkAuthorNameMehr}{Mehrdad Khodaverdi ctm546@alumni.ku.dk} % Your name
%\newcommand{\hmwkAuthorNameJonas}{Jonas Horstmann Qzj408@alumni.ku.dk} % Your name
%\newcommand{\hmwkAuthorNameVic}{Victor B. Rasmussen cwv180@alumni.ku.dk} % Your name


%----------------------------------------------------------------------------------------
%	TITLE PAGE
%----------------------------------------------------------------------------------------

\title{
\vspace{2in}
\textmd{\textbf{\hmwkClass:\ \hmwkTitle}}\\
\normalsize\vspace{0.1in}\small{Due\ on\ \hmwkDueDate}\\
%\vspace{0.1in}\large{\textit{\hmwkClassInstructor\ \hmwkClassTime}}
\vspace{3in}
}

\author{
%\textbf{\hmwkAuthorNameJonas}\\
%\textbf{\hmwkAuthorNameVic}\\
\textbf{\hmwkAuthorNameMehr}
%\date{Friday,\ October\ 2,\ 2014} % Insert date here if you want it to appear below your name
}

%----------------------------------------------------------------------------------------

\begin{document}

\maketitle

%----------------------------------------------------------------------------------------
%	TABLE OF CONTENTS
%----------------------------------------------------------------------------------------

%\setcounter{tocdepth}{1} % Uncomment this line if you don't want subsections listed in the ToC

%\newpage
%\tableofcontents
%\newpage

%----------------------------------------------------------------------------------------
%	PROBLEM 1
%----------------------------------------------------------------------------------------

% To have just one problem per page, simply put a \clearpage after each problem
\clearpage
\subsection{i5-3}

\verb|arraySort l | starts by matching if \verb|l| is an empty or an array that has only one element. If so return the empty or 1 element array.

For all other cases go through the array \verb|l| and return a new array where all the elements are smaller then the first element of \verb|l|. The result gets feeded to \verb|arraysort| till the smallest element in \verb|l| is found.

\begin{verbatim} l |> Array.filter ((>)(Array.head l)) |> arraySort\end{verbatim} 

This smallest element we place in front of the first element in \verb|l| and assign it to \verb|temp|.

\begin{verbatim} let temp = [|Array.head l|] |> Array.append \end{verbatim}

Next we start filtering l for all elements bigger or equal the first element in \verb|l| 

\begin{verbatim} l |> Array.filter ((<=)(Array.head l)) \end{verbatim}

we find the tail and feed it to \verb|arraysort| so it can get sorted as well.

\begin{verbatim} |> Array.tail |>arraySort \end{verbatim}

and one by one we attach the next smallest element behind the last smallest element.

\begin{verbatim} |> Array.append temp \end{verbatim}


\lstset{}
\begin{lstlisting}
let rec arraySort l = 
    match l with
    |[||] |[|_|] -> l
    | _ -> 
        let temp = [|Array.head l|] |> Array.append 
       	 (l |> Array.filter ((>)(Array.head l)) |> arraySort)
        l |> Array.filter ((<=)(Array.head l)) |> Array.tail 
         |>arraySort |> Array.append temp

\end{lstlisting}



%----------------------------------------------------------------------------------------
%	PROBLEM 2
%----------------------------------------------------------------------------------------

\subsection{i5-4} 

ArraySortD is a bubbleSort algorithm. Here it uses two loops to sort a list, here in this case an array and i use one for-loop and one while-loop. The for loop runs from 1 to the number of elements in the given array. And the while loop runs depending on how the array is sorted at start.

The for loop start looking at the first element then next and so on. The while loop pushes elements that are smaller then the previous element in the array to the left. So to speak to the head of the array. This is done by using a swap method.

\begin{verbatim}
	let tmp = a.[j]
	a.[j] <- a.[j-1]
	a.[j-1] <- tmp
\end{verbatim}

The function is given by: 

\lstset{}
\begin{lstlisting}
let arraySortD (a : 'a []) =
    for i = 1 to a.Length - 1 do
        let mutable j = i
        while j >= 1 && a.[j] < a.[j-1] do
            let tmp = a.[j]
            a.[j] <- a.[j-1]
            a.[j-1] <- tmp
            j <- j - 1

\end{lstlisting}




%----------------------------------------------------------------------------------------

%----------------------------------------------------------------------------------------
%	PROBLEM 3
%----------------------------------------------------------------------------------------


%----------------------------------------------------------------------------------------
%----------------------------------------------------------------------------------------
%	PROBLEM x
%----------------------------------------------------------------------------------------



%----------------------------------------------------------------------------------------

\end{document}