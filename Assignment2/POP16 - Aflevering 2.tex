%%%%%%%%%%%%%%%%%%%%%%%%%%%%%%%%%%%%%%%%%
% Programming/Coding Assignment
% LaTeX Template
%
% This template has been downloaded from:
% http://www.latextemplates.com
%
% Original author:
% Ted Pavlic (http://www.tedpavlic.com)
%
% Note:
% The \lipsum[#] commands throughout this template generate dummy text
% to fill the template out. These commands should all be removed when 
% writing assignment content.
%
% This template uses a Perl script as an example snippet of code, most other
% languages are also usable. Configure them in the "CODE INCLUSION 
% CONFIGURATION" section.
%
%%%%%%%%%%%%%%%%%%%%%%%%%%%%%%%%%%%%%%%%%

%----------------------------------------------------------------------------------------
%	PACKAGES AND OTHER DOCUMENT CONFIGURATIONS
%----------------------------------------------------------------------------------------

\documentclass{article}

\usepackage[utf8]{inputenc} %For æ ø å and other danish symbols
\usepackage{fancyhdr} % Required for custom headers
\usepackage{lastpage} % Required to determine the last page for the footer
\usepackage{extramarks} % Required for headers and footers
\usepackage[usenames,dvipsnames]{color} % Required for custom colors
\usepackage{graphicx} % Required to insert images
\usepackage{listings} % Required for insertion of code
\usepackage[]{algorithm2e} % For algortihms 
\usepackage{courier} % Required for the courier font
\usepackage{lipsum} % Used for inserting dummy 'Lorem ipsum' text into the template
\usepackage{color}
\usepackage{amsmath}
\usepackage{amssymb}
\usepackage{amsthm}
\usepackage{upquote,textcomp}

\usepackage[hidelinks]{hyperref} % For URL ref
\usepackage{xcolor}
\hypersetup{
	colorlinks,
	linkcolor={red!50!black},
	citecolor={blue!50!black},
	urlcolor={blue!80!black}
}

\graphicspath{ {images/} } %all images are in the folder images

% Margins
\topmargin=-0.45in
\evensidemargin=0in
\oddsidemargin=0in
\textwidth=6.5in
\textheight=9.0in
\headsep=0.25in

\linespread{1.1} % Line spacing

% Set up the header and footer
\pagestyle{fancy}
%\lhead{\hmwkAuthorNameMehr} % Top left header
\chead{\hmwkClass: \hmwkTitle} % Top center head
\rhead{\firstxmark} % Top right header
\lfoot{\lastxmark} % Bottom left footer
\cfoot{} % Bottom center footer
\rfoot{Page\ \thepage\ of\ \protect\pageref{LastPage}} % Bottom right footer
\renewcommand\headrulewidth{0.4pt} % Size of the header rule
\renewcommand\footrulewidth{0.4pt} % Size of the footer rule

\setlength\parindent{0pt} % Removes all indentation from paragraphs

%----------------------------------------------------------------------------------------
%	DOCUMENT STRUCTURE COMMANDS
%	Skip this unless you know what you're doing
%----------------------------------------------------------------------------------------

% Header and footer for when a page split occurs within a problem environment
\newcommand{\enterProblemHeader}[1]{
\nobreak\extramarks{#1}{#1 continued on next page\ldots}\nobreak
\nobreak\extramarks{#1 (continued)}{#1 continued on next page\ldots}\nobreak
}

% Header and footer for when a page split occurs between problem environments
\newcommand{\exitProblemHeader}[1]{
\nobreak\extramarks{#1 (continued)}{#1 continued on next page\ldots}\nobreak
\nobreak\extramarks{#1}{}\nobreak
}

\setcounter{secnumdepth}{0} % Removes default section numbers
\newcounter{homeworkProblemCounter} % Creates a counter to keep track of the number of problems

\newcommand{\homeworkProblemName}{}
\newenvironment{homeworkProblem}[1][1.8 Exercises \arabic{homeworkProblemCounter}]{ % Makes a new environment called homeworkProblem which takes 1 argument (custom name) but the default is "Problem #"
\stepcounter{homeworkProblemCounter} % Increase counter for number of problems
\renewcommand{\homeworkProblemName}{#1-1} % Assign \homeworkProblemName the name of the problem
\section{\homeworkProblemName} % Make a section in the document with the custom problem count
\enterProblemHeader{\homeworkProblemName} % Header and footer within the environment
}{
\exitProblemHeader{\homeworkProblemName} % Header and footer after the environment
}

%----------------------------------------------------------------------------------------
%	NAME AND CLASS SECTION
%----------------------------------------------------------------------------------------

\newcommand{\hmwkTitle}{Ugeseddel\ \#1} % Assignment title
\newcommand{\hmwkDueDate}{Wednesday,\ September\ 14,\ 2016} % Due date
\newcommand{\hmwkClass}{Programmering og problemløåsning 5100-B1-2E16} % Course/class
%\newcommand{\hmwkClassTime}{09:15am} % Class/lecture time
%\newcommand{\hmwkClassInstructor}{Jones} % Teacher/lecturer
\newcommand{\hmwkAuthorNameMehr}{Mehrdad Khodaverdi ctm546@alumni.ku.dk} % Your name
%\newcommand{\hmwkAuthorNameJonas}{Jonas Horstmann Qzj408@alumni.ku.dk} % Your name
%\newcommand{\hmwkAuthorNameVic}{Victor B. Rasmussen cwv180@alumni.ku.dk} % Your name


%----------------------------------------------------------------------------------------
%	TITLE PAGE
%----------------------------------------------------------------------------------------

\title{
\vspace{2in}
\textmd{\textbf{\hmwkClass:\ \hmwkTitle}}\\
\normalsize\vspace{0.1in}\small{Due\ on\ \hmwkDueDate}\\
%\vspace{0.1in}\large{\textit{\hmwkClassInstructor\ \hmwkClassTime}}
\vspace{3in}
}

\author{
%\textbf{\hmwkAuthorNameJonas}\\
%\textbf{\hmwkAuthorNameVic}\\
\textbf{\hmwkAuthorNameMehr}
%\date{Friday,\ October\ 2,\ 2014} % Insert date here if you want it to appear below your name
}

%----------------------------------------------------------------------------------------

\begin{document}

\maketitle

%----------------------------------------------------------------------------------------
%	TABLE OF CONTENTS
%----------------------------------------------------------------------------------------

%\setcounter{tocdepth}{1} % Uncomment this line if you don't want subsections listed in the ToC

%\newpage
%\tableofcontents
%\newpage

%----------------------------------------------------------------------------------------
%	PROBLEM 1
%----------------------------------------------------------------------------------------

% To have just one problem per page, simply put a \clearpage after each problem
\clearpage
\subsection{2i.0}

Betragt EBNF’en:\\

charLiteral = ?any unicode codepoint?;\\
stringLiteral = \verb|'"'|, {charLiteral}, \verb|'"'|;\\
operator = \verb|'+'|;\\
expression = stringLiteral $|$ stringLiteral, operator, expression; \\

\textbf{1.} Opskriv 3 forskellige gyldige expressions udelukkende ved brug af tokenerne
expressions, operator og stringLiteral:\\

\begin{tabular}{lp{10cm}}
	expression = stringLiteral & (1) \\
	expression = stringLiteral, operator, stringLiteral & (2)\\
	expression = stringLiteral, operator, expression & (3)
\end{tabular}\\\\


\textbf{2.} Giv derefter eksempler på tilsvarende sekvenser, hvor tokenerne er erstattet med terminaler.\\

For charLiteral = a \\

\begin{tabular}{lp{10cm}}
	\verb|'"a"'| = \verb|'"a"'| & (1) \\
	\verb|'"a"+"a"'| = \verb|'"a"'| \verb|'+'| \verb|'"a"'| & (2)\\
	\verb|'"a"+"a"+"a"'| = \verb|'"a"'| \verb|'+'| \verb|'"a"+"a"'| & (3)
\end{tabular}\\\\

\textbf{3.} Giv et eksempel på en sekvens, som ikke er gyldig i ovenstående EBNF.\\

For charLiteral = a \\

\begin{tabular}{lp{10cm}}
	expression $\neq$ operator \\
	expression $\neq$ \verb|'+'| \\\\
	expression $\neq$ stringLiteral, operator, operator \\
	expression $\neq$ \verb|'"a"'| \verb|'+'| \verb|'+'| \\\\
	expression $\neq$ stringLiteral, stringLiteral, expression \\
	expression $\neq$ \verb|'"a"'| \verb|'"a"'| \verb|'"a"+"a"'| 
\end{tabular}\\




%----------------------------------------------------------------------------------------
%	PROBLEM 2
%----------------------------------------------------------------------------------------

\subsection{2i.1}

Udfyld følgende tabel

\begin{table}[h]
	\centering
	\begin{tabular}{|c|c|c|c|}
		\hline
		Decimal & Binær & Heximal & Oktal  \\ \cline{1-4}
		10 & 1010 & A & 12 \\ \cline{1-4}
		21 & 10101 & 15 & 25  \\ \cline{1-4}
		63 & 111111 & 3F & 77 \\ \cline{1-4}
		63 & 111111 & 3F & 77 \\ \cline{1-4}
	\end{tabular}
\end{table}

\clearpage

\textbf{Første række:}

\underline{Decimal 10 til binær:}
\begin{align*}
	\frac{10}{2} &= 5 \hphantom{10} \text{Rest 0}\\
	\frac{5}{2} &= 2 \hphantom{10} \text{Rest 1}\\
	\frac{2}{2} &= 1 \hphantom{10} \text{Rest 0}\\
	\frac{1}{2} &= 1 \hphantom{10} \text{Rest 1}\\
\end{align*}

Ergo $(10)_{10}$ = $(1010)_2$\\

Det er givet at 

\begin{table}[h]
	\centering
	\begin{tabular}{|c|c|c|c|c|c|c|c|c|c|c|c|c|c|c|c|c|c|}
		\hline
		\textbf{Dec}&0&1&2&3&4&5&6&7&8&9&10&11&12&13&14&15\\ \cline{1-17}
		\textbf{Hex}&0&1&2&3&4&5&6&7&8&9&A&B&C&D&E&F\\ \cline{1-17}
	\end{tabular}
\end{table}

Derfor må $(10)_{10}$ = $(A)_{Hex}$\\

\underline{Decimal 10 til Oktal:}
$$
10\mod 8 = 2 \hphantom{10} \text{hvilket man regner ved:} \frac{10}{8} = \underline{\textbf{1}}.25 \hphantom{10} \Rightarrow 8^{\underline{\textbf{1}}} + 2 = 10 \hphantom{10} \text{altså rest 2}\\
$$
Vores Octal nummer er nu ??2 og vi gentager igen fra før:
$$
1\mod 8 = 1 \hphantom{10} \text{hvilket man regner ved:} \frac{1}{8} = \underline{\textbf{0}}.125 \hphantom{10} \Rightarrow 8^{\underline{\textbf{0}}} + 1 = 1 \hphantom{10} \text{altså rest 1}\\
$$
Ergo tallet er 12 .\\

Derfor må $(10)_{10}$ = $(12)_{8}$\\

\textbf{Anden række:}\\
\underline{Binær 10101 til decimal:\\}

\begin{table}[h]
	\centering
	\begin{tabular}{ccccc}
		$1$&$0$&$1$&$0$&$1$\\ 
		$2^4$&$+2^3$&$+2^2$&$+2^1$&$+2^0$\\ 
	\end{tabular}
\end{table}

10101 betyder altså $2^0 + 2^2 + 2^4 = 21$ \\

Ergo $(10101)_{2}$ = $(21)_{10}$\\

\textbf{10101 til Hex:}

Vi udvider vores tabel gennem samme øvelse som vist oven over.:
\begin{table}[!htbp]
	\setlength\tabcolsep{4pt}
	\begin{tabular}{|c|c|c|c|c|c|c|c|c|c|c|c|c|c|c|c|c|c|}
		\hline
		\textbf{Dec}&0&1&2&3&4&5&6&7&8&9&10&11&12&13&14&15\\ \cline{1-17}
		\textbf{Bin}&0000&0001&0010&0011&0100&0101&0110&0111&1000&1001&1010&1011&1100&1101&1110&1111\\ \cline{1-17}
		\textbf{Hex}&0&1&2&3&4&5&6&7&8&9&A&B&C&D&E&F\\ \cline{1-17}
	\end{tabular}
\end{table}

\clearpage
Vi ser igen på 10101. Da vi ved at 1 Hex ækvivalent med 4 bit deler vi det binær tal op i grupper af 4. 10101 skrives som 0001 0101 og fra tabellen har vi:
\begin{table}[h]
	\centering
	\begin{tabular}{cc}
		$0001$&$0101$\\ 
		$1$&$5$\\ 
	\end{tabular}
\end{table}

Ergo $(10101)_{2}$ = $(15)_{Hex}$\\

\underline{10101 til Oktal:}

Vi udvider vores tabel nu med oktal:


\begin{table}[!htbp]
	\caption{Konverterings tabel}
	\setlength\tabcolsep{4pt}
	\begin{tabular}{|c|c|c|c|c|c|c|c|c|c|c|c|c|c|c|c|c|c|}
		\hline
		\textbf{Dec}&0&1&2&3&4&5&6&7&8&9&10&11&12&13&14&15\\ \cline{1-17}
		\textbf{Bin}&0000&0001&0010&0011&0100&0101&0110&0111&1000&1001&1010&1011&1100&1101&1110&1111\\ \cline{1-17}
		\textbf{Hex}&0&1&2&3&4&5&6&7&8&9&A&B&C&D&E&F\\ \cline{1-17}
		\textbf{Oktal}&0&1&2&3&4&5&6&7\\ \cline{1-9}
	\end{tabular}
\end{table}

Vi ser på 10101. Da vi ved at 1 oktal er ækvivalent med 3 bit deler vi det binær tal op i grupper af 3. 10101 skrives som 010 101 og ved brug af tabellen får:

\begin{table}[!htbp]
	\centering
	\begin{tabular}{cc}
		$010$&$101$\\ 
		$2$&$5$\\ 
	\end{tabular}
\end{table}

Ergo $(10101)_{2}$ = $(25)_{8}$\\

\textbf{Tredje række:}

\underline{Hex 3F til Bin og til oktal:}

Omskriver hex 3F til bin: 3 \textrightarrow 0011 og F \textrightarrow 1111. Ergo $(3\text{F})_{Hex}$ = $(0011 1111)_{2}$\\

Nu deler vi det binær tal op i grupper af 3 og får 000 111 111. Fra tabellen har vi:

\begin{table}[!htbp]
	\centering
	\begin{tabular}{ccc}
		$000$&$111$&$111$\\ 
		$0$&$7$&$7$\\ 
	\end{tabular}
\end{table}

Ergo $(3\text{F})_{Hex}$ = $(77)_{8}$\\

\underline{Hex til decimal:}

3F = $(3*16^1)+(15*16^0) = 48 + 15 = 63$ \\
Ergo $(3\text{F})_{Hex}$ = $(63)_{10}$\\

\textbf{Fjerde række:}

\underline{oktal 77 til bin også til hex}

Vi ser hurtigt fra tabel 1 at tallet 77 kan skrives som binært:
\begin{table}[!htbp]
	\centering
	\begin{tabular}{cc}
		$7$&$7$\\ 
		$111$&$111$\\ 
	\end{tabular}
\end{table}

Ergo $(77)_{8}$ = $(111 111)_{2}$\\ og igen kan det grupperes som 0011 1111 hvilket giver henholdsvis tallet 3 og F, ergo 3F




%----------------------------------------------------------------------------------------

%----------------------------------------------------------------------------------------
%	PROBLEM 3
%----------------------------------------------------------------------------------------

\subsection{2i.2}

Først defineres \textit{\textbf{a}} som "Hello World"

Derefter printer jeg de første fire char i a, for til sidst at printe fra den sjette char til slut. 

\lstset{}
\begin{lstlisting}
	> let a = "Hello World";;
	
	val a : string = "Hello World"
	
	> a.[..4];;
	val it : string = "Hello"
	> a.[6..];;
	val it : string = "World"
	
	> a.[6..]+", "+a.[..4]+"!";;
	val it : string = "World, Hello!"

\end{lstlisting}




%----------------------------------------------------------------------------------------
%----------------------------------------------------------------------------------------
%	PROBLEM x
%----------------------------------------------------------------------------------------



%----------------------------------------------------------------------------------------

\end{document}