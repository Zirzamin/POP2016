\subsection{2i.1}

Udfyld følgende tabel

\begin{table}[h]
	\centering
	\begin{tabular}{|c|c|c|c|}
		\hline
		Decimal & Binær & Heximal & Oktal  \\ \cline{1-4}
		10 & 1010 & A & 12 \\ \cline{1-4}
		21 & 10101 & 15 & 25  \\ \cline{1-4}
		63 & 111111 & 3F & 77 \\ \cline{1-4}
		63 & 111111 & 3F & 77 \\ \cline{1-4}
	\end{tabular}
\end{table}

\clearpage

\textbf{Første række:}

\underline{Decimal 10 til binær:}
\begin{align*}
	\frac{10}{2} &= 5 \hphantom{10} \text{Rest 0}\\
	\frac{5}{2} &= 2 \hphantom{10} \text{Rest 1}\\
	\frac{2}{2} &= 1 \hphantom{10} \text{Rest 0}\\
	\frac{1}{2} &= 1 \hphantom{10} \text{Rest 1}\\
\end{align*}

Ergo $(10)_{10}$ = $(1010)_2$\\

Det er givet at 

\begin{table}[h]
	\centering
	\begin{tabular}{|c|c|c|c|c|c|c|c|c|c|c|c|c|c|c|c|c|c|}
		\hline
		\textbf{Dec}&0&1&2&3&4&5&6&7&8&9&10&11&12&13&14&15\\ \cline{1-17}
		\textbf{Hex}&0&1&2&3&4&5&6&7&8&9&A&B&C&D&E&F\\ \cline{1-17}
	\end{tabular}
\end{table}

Derfor må $(10)_{10}$ = $(A)_{Hex}$\\

\underline{Decimal 10 til Oktal:}
$$
10\mod 8 = 2 \hphantom{10} \text{hvilket man regner ved:} \frac{10}{8} = \underline{\textbf{1}}.25 \hphantom{10} \Rightarrow 8^{\underline{\textbf{1}}} + 2 = 10 \hphantom{10} \text{altså rest 2}\\
$$
Vores Octal nummer er nu ??2 og vi gentager igen fra før:
$$
1\mod 8 = 1 \hphantom{10} \text{hvilket man regner ved:} \frac{1}{8} = \underline{\textbf{0}}.125 \hphantom{10} \Rightarrow 8^{\underline{\textbf{0}}} + 1 = 1 \hphantom{10} \text{altså rest 1}\\
$$
Ergo tallet er 12 .\\

Derfor må $(10)_{10}$ = $(12)_{8}$\\

\textbf{Anden række:}\\
\underline{Binær 10101 til decimal:\\}

\begin{table}[h]
	\centering
	\begin{tabular}{ccccc}
		$1$&$0$&$1$&$0$&$1$\\ 
		$2^4$&$+2^3$&$+2^2$&$+2^1$&$+2^0$\\ 
	\end{tabular}
\end{table}

10101 betyder altså $2^0 + 2^2 + 2^4 = 21$ \\

Ergo $(10101)_{2}$ = $(21)_{10}$\\

\textbf{10101 til Hex:}

Vi udvider vores tabel gennem samme øvelse som vist oven over.:
\begin{table}[!htbp]
	\setlength\tabcolsep{4pt}
	\begin{tabular}{|c|c|c|c|c|c|c|c|c|c|c|c|c|c|c|c|c|c|}
		\hline
		\textbf{Dec}&0&1&2&3&4&5&6&7&8&9&10&11&12&13&14&15\\ \cline{1-17}
		\textbf{Bin}&0000&0001&0010&0011&0100&0101&0110&0111&1000&1001&1010&1011&1100&1101&1110&1111\\ \cline{1-17}
		\textbf{Hex}&0&1&2&3&4&5&6&7&8&9&A&B&C&D&E&F\\ \cline{1-17}
	\end{tabular}
\end{table}

\clearpage
Vi ser igen på 10101. Da vi ved at 1 Hex ækvivalent med 4 bit deler vi det binær tal op i grupper af 4. 10101 skrives som 0001 0101 og fra tabellen har vi:
\begin{table}[h]
	\centering
	\begin{tabular}{cc}
		$0001$&$0101$\\ 
		$1$&$5$\\ 
	\end{tabular}
\end{table}

Ergo $(10101)_{2}$ = $(15)_{Hex}$\\

\underline{10101 til Oktal:}

Vi udvider vores tabel nu med oktal:


\begin{table}[!htbp]
	\caption{Konverterings tabel}
	\setlength\tabcolsep{4pt}
	\begin{tabular}{|c|c|c|c|c|c|c|c|c|c|c|c|c|c|c|c|c|c|}
		\hline
		\textbf{Dec}&0&1&2&3&4&5&6&7&8&9&10&11&12&13&14&15\\ \cline{1-17}
		\textbf{Bin}&0000&0001&0010&0011&0100&0101&0110&0111&1000&1001&1010&1011&1100&1101&1110&1111\\ \cline{1-17}
		\textbf{Hex}&0&1&2&3&4&5&6&7&8&9&A&B&C&D&E&F\\ \cline{1-17}
		\textbf{Oktal}&0&1&2&3&4&5&6&7\\ \cline{1-9}
	\end{tabular}
\end{table}

Vi ser på 10101. Da vi ved at 1 oktal er ækvivalent med 3 bit deler vi det binær tal op i grupper af 3. 10101 skrives som 010 101 og ved brug af tabellen får:

\begin{table}[!htbp]
	\centering
	\begin{tabular}{cc}
		$010$&$101$\\ 
		$2$&$5$\\ 
	\end{tabular}
\end{table}

Ergo $(10101)_{2}$ = $(25)_{8}$\\

\textbf{Tredje række:}

\underline{Hex 3F til Bin og til oktal:}

Omskriver hex 3F til bin: 3 \textrightarrow 0011 og F \textrightarrow 1111. Ergo $(3\text{F})_{Hex}$ = $(0011 1111)_{2}$\\

Nu deler vi det binær tal op i grupper af 3 og får 000 111 111. Fra tabellen har vi:

\begin{table}[!htbp]
	\centering
	\begin{tabular}{ccc}
		$000$&$111$&$111$\\ 
		$0$&$7$&$7$\\ 
	\end{tabular}
\end{table}

Ergo $(3\text{F})_{Hex}$ = $(77)_{8}$\\

\underline{Hex til decimal:}

3F = $(3*16^1)+(15*16^0) = 48 + 15 = 63$ \\
Ergo $(3\text{F})_{Hex}$ = $(63)_{10}$\\

\textbf{Fjerde række:}

\underline{oktal 77 til bin også til hex}

Vi ser hurtigt fra tabel 1 at tallet 77 kan skrives som binært:
\begin{table}[!htbp]
	\centering
	\begin{tabular}{cc}
		$7$&$7$\\ 
		$111$&$111$\\ 
	\end{tabular}
\end{table}

Ergo $(77)_{8}$ = $(111 111)_{2}$\\ og igen kan det grupperes som 0011 1111 hvilket giver henholdsvis tallet 3 og F, ergo 3F


